%%% A template for a simple PDF/A file like a stand-alone abstract of the thesis.

\documentclass[12pt]{report}

\usepackage[a4paper, hmargin=1in, vmargin=1in]{geometry}
\usepackage[a-2u]{pdfx}
\usepackage[utf8]{inputenc}
\usepackage[T1]{fontenc}
\usepackage{lmodern}
\usepackage{textcomp}

\begin{document}

%% Do not forget to edit abstract.xmpdata.

Přenos energie v molekulárních agregátech je obecně obtížné popsat jednoduchým, ale účinným způsobem. Často dochází ke kompromisu mezi přesností simulovaných výsledků a úrovní pochopení odpovídajících fyzikálních procesů. Pro pochopení vývoje systému s elektronovými stupni volnosti je také nutné porozumět vlivu systému na vývoj lázně. K získání vhledu do vývoje lázně zavádíme exaktní faktorizaci elementů matice hustoty reprezentující provázaný stav lázně a systému. Této faktorizace využijeme k odvození iterativních řídících rovnic. Iterativní popis vývoje lázně pak použijeme k odvození korigovaného paměťového jádra s korelačními funkcemi v lokální bázi s předpokladem lineárních harmonických oscilátorů jako módů lázně. Tento přístup se pokouší vylepšit stávající poruchové řídící rovnice v režimu slabé interakce mezi systémem a lázní. K posouzení dosaženého vylepšení použijeme teorii na systémy s konečnou lázní malé velikosti.

\end{document}
